\documentclass[12pt, a4]{article}
\usepackage[utf8]{inputenc}
\usepackage[T1]{fontenc} 
\usepackage{mathptmx}
\usepackage[document]{ragged2e}
\usepackage{color}
\usepackage{listings}
\usepackage{hyperref}
\usepackage{graphicx}
\graphicspath{ {./} }
\usepackage{textcomp}

\lstalias[]{ES6}[ECMAScript2015]{JavaScript}

\lstdefinelanguage{JavaScript}{
	morekeywords=[1]{break, continue, delete, else, for, function, if, in,
		new, return, this, typeof, var, void, while, with},
	% Literals, primitive types, and reference types.
	morekeywords=[2]{false, null, true, boolean, number, undefined,
		Array, Boolean, Date, Math, Number, String, Object},
	% Built-ins.
	morekeywords=[3]{eval, parseInt, parseFloat, escape, unescape},
	sensitive,
	morecomment=[s]{/*}{*/},
	morecomment=[l]//,
	morecomment=[s]{/**}{*/}, % JavaDoc style comments
	morestring=[b]',
	morestring=[b]"
}[keywords, comments, strings]

\lstdefinelanguage[ECMAScript2015]{JavaScript}[]{JavaScript}{
	morekeywords=[1]{await, async, case, catch, class, const, default, do,
		enum, export, extends, finally, from, implements, import, instanceof,
		let, static, super, switch, throw, try},
	morestring=[b]` % Interpolation strings.
}

\definecolor{mediumgray}{rgb}{0.3, 0.4, 0.4}
\definecolor{mediumblue}{rgb}{0.0, 0.0, 0.8}
\definecolor{forestgreen}{rgb}{0.13, 0.55, 0.13}
\definecolor{darkviolet}{rgb}{0.58, 0.0, 0.83}
\definecolor{royalblue}{rgb}{0.25, 0.41, 0.88}
\definecolor{crimson}{rgb}{0.86, 0.8, 0.24}

\lstdefinestyle{JSES6Base}{
	backgroundcolor=\color{white},
	basicstyle=\ttfamily,
	breakatwhitespace=false,
	breaklines=false,
	captionpos=b,
	columns=fullflexible,
	commentstyle=\color{mediumgray}\upshape,
	emph={},
	emphstyle=\color{crimson},
	extendedchars=true,  % requires inputenc
	fontadjust=true,
	frame=single,
	identifierstyle=\color{black},
	keepspaces=true,
	keywordstyle=\color{mediumblue},
	keywordstyle={[2]\color{darkviolet}},
	keywordstyle={[3]\color{royalblue}},
	numbers=left,
	numbersep=5pt,
	numberstyle=\tiny\color{black},
	rulecolor=\color{black},
	showlines=true,
	showspaces=false,
	showstringspaces=false,
	showtabs=false,
	stringstyle=\color{forestgreen},
	tabsize=2,
	title=\lstname,
	upquote=true  % requires textcomp
}

\lstdefinestyle{JavaScript}{
	language=JavaScript,
	style=JSES6Base
}
\lstdefinestyle{ES6}{
	language=ES6,
	style=JSES6Base
}

\begin{document}
	\begin{center}
	\MakeUppercase{\textbf{Rīgas TehniskĀ universitĀte}}
	\newline
	\MakeUppercase{\textbf{Datorzinātnes un informācijas tehnoloģijas fakultāte}}
	\vspace{5cm}
	\newline
	\textbf{Datorgrafikas un attēlu apstrādes pamati}
	\newline
	\textbf{Praktiskais darbs \#3}
	\newline
	Bezje liknes veidošanas algoritms
	\end{center}
\vspace{7cm}
\begin{flushright}
	\textbf{
	D I T F
	\newline
	1. kurss 1. grupa
	\newline
	Edgars Jānis Kuzmanis
	\newline
	191RDB175
}
\end{flushright}
\newpage
Darba uzdevums
\begin{itemize}
	\item Izveidot programmu, kas realizētu Bezjē (Bezier) līknes algoritmu, ar noteikto kontrolpunktu skaitu (sk. pēc varianta*).
	\item Realizēt programmā iespēju izvēlēties kontrolpunktu skaitu no 3 līdz 12 un izveidot Bezjē līkni atbilstoši izvēlētajam kontrolpunktu skaitam.  Izmantot Bezjē līknes formulu, kur tiek pielietots faktoriāla izskaitļojums
	\item Realizēt programmā iespēju ievadīt jebkuru Bezjē līknes precizitāti (parametra t solis)
	\item Ortusā jāielādē arī visus projekta failus, kas ir saistīti ar programmu, saspiestā veidā *.arj, *.zip vai *.7z formātā.
\end{itemize}

\newpage
Šis mājasdarbs ir taisīts ar React (JSX) un kaut gan sintakse ir līdzīga Java, vairākas lietas atšķiras, vislielākā būdama struktūra.
Šo projektu var palaist lokāli, bet uz windows tas būs grūtāk nekā uz UNIX balstītas operētājsistēmas.
Šeit var redzēt kodu galvenajam komponentam (failam), kurā viss notiek, bet labāk kodu var apskatīt \href{https://github.com/ragem/BezierDrawer/blob/master/src/Components/BezierDrawer/BezierDrawer.component.js}{šeit}
(https://github.com).
Tur būs pilnais programmas kods gan ar kārtīgām indentācijām un krāsām.
\newline
Demo link \url{https://ragem.github.io/BezierDrawer/} 
\newline
(iesaku step precision likt uz 0.01 lai skaistaka linija)
\newline
(verot vaļā šo kā hyperlink viņš dažreiz nestradā, iesaku iekopēt URL un ielīmēt browserī)
\begin{lstlisting}[style=ES6, caption={BezierDrawer.component.js}]
import React, { PureComponent } from 'react';

class BezierDrawer extends PureComponent {
constructor() {
super();
this.controlpoints = React.createRef();
this.canvas = React.createRef();
this.precision = React.createRef();
}

onDrawClick() {
const controlpoints = this.controlpoints.current.value;
this.drawCurve(this.generateRandomCoords(controlpoints));
}

bezier(t, plist) {
const order = plist.length - 1;

let y = 0;
let x = 0;

for (let i = 0; i <= order; i++) {
// eslint-disable-next-line no-restricted-properties
x += (this.binom(order, i) * Math.pow((1 - t), (order - i)) * Math.pow(t, i) * (plist[i].x));
// eslint-disable-next-line no-restricted-properties
y += (this.binom(order, i) * Math.pow((1 - t), (order - i)) * Math.pow(t, i) * (plist[i].y));
}

return {
x, y
};
}

binom(n, k) {
let coeff = 1;
for (let i = n - k + 1; i <= n; i++) coeff *= i;
for (let i = 1; i <= k; i++) coeff /= i;
return coeff;
}

generateRandomCoords(count) {
const coords = [];
for (let i = 0; i < count; i++) {
const x = Math.floor(Math.random() * 500);
const y = Math.floor(Math.random() * 500);
coords.push({ x, y });
}
return coords;
}

drawCurve(plist) {
const ctx = this.canvas.current.getContext('2d');
const canvas = this.canvas.current.getBoundingClientRect();
ctx.clearRect(0, 0, canvas.width, canvas.height);
const accuracy = parseFloat(this.precision.current.value);
ctx.beginPath();
ctx.moveTo(plist[0].x, plist[0].y);

// eslint-disable-next-line no-restricted-syntax
for (const p in plist) {
if (p) {
ctx.fillText(p, plist[p].x + 5, plist[p].y - 5);
ctx.fillRect(plist[p].x - 5, plist[p].y - 5, 2, 2);
}
}

for (let i = 0; i < 1; i += accuracy) {
const { x, y } = this.bezier(i, plist);
ctx.lineTo(x, y);
}

ctx.stroke();
ctx.closePath();
}

renderCanvas() {
return (
<canvas
ref={ this.canvas }
className="Canvas"
width={ 500 }
height={ 500 }
/>
);
}

renderControls() {
return (
<div className="controls">
<label htmlFor="controlpoints">
Number of control points
<input type="number" ref={ this.controlpoints } id="controlpoints" />
</label>
<label htmlFor="precision">
step precision
<input type="number" ref={ this.precision } id="precision" step="0.001" />
</label>
<button onClick={ () => this.onDrawClick() }>
Draw
</button>
</div>
);
}

render() {
return (
<div>
{ this.renderCanvas() }
{ this.renderControls() }
</div>
);
}
}

export default BezierDrawer;

\end{lstlisting}
\newpage
\includegraphics[width=\textwidth,height=\textheight,keepaspectratio]{md3}


\end{document}